\documentclass[journal,12pt,twocolumn]{IEEEtran}
\usepackage{tikz}
\usepackage{amsmath}
\usepackage{amssymb}
\pagestyle{empty}
\usepackage{setspace}
\singlespacing
\usepackage{caption}
\captionsetup{justification=centering}
\usepackage{amsthm}
\usepackage{amssymb,amsmath}


\begin{document}
\newcommand{\myvec}[1]{\ensuremath{\begin{pmatrix}#1\end{pmatrix}}}
\newcommand{\cmyvec}[1]{\ensuremath{\begin{pmatrix*}[c]#1\end{pmatrix*}}}
\providecommand{\norm}[1]{\lVert#1\rVert}
\newcommand{\mydet}[1]{\ensuremath{\begin{vmatrix}#1\end{vmatrix}}}
\newcommand{\proj}[2]{\textbf{proj}_{\vec{#1}}\vec{#2}}
\newcommand{\abs}[1]{\left\lvert#1\right\rvert}
\newcommand{\RNum}[1]{\uppercase\expandafter{\romannumeral #1\relax}}
\newcommand{\Rnum}[1]{\lowercase\expandafter{\romannumeral #1\relax}}
\let\StandardTheFigure\thefigure
\let\vec\mathbf

\title{
BASICS OF PROGRAMMING

ASSIGNMENT - 2
}
\author{ LAKSHMI GAYATHRI GUDIPUDI - SM21MTECH11001}
\maketitle
\newpage
\bigskip
\renewcommand{\thefigure}{\theenumi}
\bibliographystyle{IEEEtran}
\section*{ Chapter \RNum{3} Ex-\RNum{4} Q-5}
\noindent
Find the condition that the lines
$$\mathbf{y+t_i x=2at_i+at_i^3=0}$$
where i=1,2,3, are concurrent. 
$$\mathbf{l_1=t_1 x + y - 2at_1 -at_1^3=0}$$
$$\mathbf{l_2=t_2 x + y - 2at_2 -at_2^3=0}$$
$$\mathbf{l_3=t_3 x + y - 2at_3 -at_3^3=0}$$
\noindent
\section*{\textbf{Solution}}
\noindent
Considering coefficients of three lines in vector form :
\begin{align}
   \mathbf{\vec{L_1}=\myvec{a_1\\b_1\\c_1}=\myvec{t_1\\1\\-2at_1-at_1^3}}\\
   \mathbf{\vec{L_2}=\myvec{a_2\\b_2\\c_2}=\myvec{t_2\\1\\- 2at_2 -at_2^3}}\\
   \mathbf{\vec{L_3}=\myvec{a_3\\b_3\\c_3}=\myvec{t_3\\1\\- 2at_3 -at_3^3}}\\
\end{align}
Now for three lines $$\mathbf{a_ix+b_iy+c=0}$$ where i=1,2,3 to be concurrent,
\begin{align}
\mathbf{\Delta=\begin{vmatrix}
\mathbf{L_1}^\intercal\\\mathbf{L_2}^\intercal\\\mathbf{L_3}^\intercal\\
\end{vmatrix}=
\begin{vmatrix}
a_1&b_1&c_1\\
a_2&b_2&c_2\\
a_3&b_3&c_3\\
\end{vmatrix}=0}
\end{align}
\begin{align}
\mathbf{\Delta=
\begin{vmatrix}
t_1&1&-2at_1-at_1^3\\
t_2&1&-2at_2-at_2^3\\
t_3&1&-2at_3-at_3^3\\
\end{vmatrix}=0}
\end{align}
On Expanding the determinant we get 
\begin{equation}
\begin{split}
t_1(-2at_3 -at_3^3 +2at_2+at_2^3)\\
-t_2(-2at_3 -at_3^3 +2at_1+at_1^3)\\
+t_3(-2at_2 -at_2^3 +2at_1+at_1^3)=0\\
\end{split}
\end{equation}

\begin{equation}
\begin{split}
t_1(2a(t_2-t_3)+a(t_2^3-t_3^3))\\
-t_2(2a(t_1-t_3)+a(t_1^3-t_3^3))\\
-t_3(2a(t_1-t_2)+a(t_1^3-t_2^3))=0\\
\end{split}
\end{equation}
On further simplifying we get
\begin{equation}
\begin{split}
2at_1t_2-2at_1t_3+at_1t_2^3-at_1t_3^3\\
-2at_1t_2+2at_2t_3-at_2t_1^3-at_2t_3^3\\
+2at_1t_3-2at_2t_3+at_3t_1^3-at_3t_2^3=0\\
\end{split}
\end{equation}

\begin{equation}
\begin{split}
t_1t_2^3-t_1t_3^3-t_2t_1^3-t_2t_3^3\\
+t_3t_1^3-t_3t_2^3=0\\
\end{split}
\end{equation}

\begin{equation}
\begin{split}
t_1^3(t_2-t_3)+t_2^3(t_3-t_1)+t_3^3(t_1-t_2)=0
\end{split}
\label{eq:1}
\end{equation}

Equation \eqref{eq:1} can be written as 
\begin{equation}
\begin{split}
\sum{t_1^3(t_2-t_3)}=0
\label{eq:2}
\end{split}
\end{equation}
Therefore Equation \eqref{eq:2} represents the condition for the three lines to be concurrent.
\end{document}

