\documentclass[journal,12pt,twocolumn]{IEEEtran}
\usepackage{tikz}
\usepackage{amsmath}
\usepackage{amssymb}
\pagestyle{empty}
\usepackage{setspace}
\singlespacing
\usepackage{caption}
\captionsetup{justification=centering}
\usepackage{amsthm}
\usepackage{mathtools} 
\usepackage{extarrows}
\usepackage{amssymb,amsmath}


\begin{document}
\newcommand{\myvec}[1]{\ensuremath{\begin{pmatrix}#1\end{pmatrix}}}
\newcommand{\cmyvec}[1]{\ensuremath{\begin{pmatrix*}[c]#1\end{pmatrix*}}}
\providecommand{\norm}[1]{\lVert#1\rVert}
\newcommand{\mydet}[1]{\ensuremath{\begin{vmatrix}#1\end{vmatrix}}}
\newcommand{\proj}[2]{\textbf{proj}_{\vec{#1}}\vec{#2}}
\newcommand{\abs}[1]{\left\lvert#1\right\rvert}
\newcommand{\RNum}[1]{\uppercase\expandafter{\romannumeral #1\relax}}
\newcommand{\Rnum}[1]{\lowercase\expandafter{\romannumeral #1\relax}}
\let\StandardTheFigure\thefigure
\let\vec\mathbf

\title{
BASICS OF PROGRAMMING

ASSIGNMENT - 2
}
\author{ LAKSHMI GAYATHRI GUDIPUDI - SM21MTECH11001}
\maketitle
\newpage
\bigskip
\renewcommand{\thefigure}{\theenumi}
\bibliographystyle{IEEEtran}
\section*{ Chapter \RNum{3} Ex-\RNum{4} Q-5}
\noindent
Find the condition that the lines
$$\mathbf{y+t_i x=2at_i+at_i^3}$$
where i=1,2,3, are concurrent. 
$$\mathbf{t_1 x + y = 2at_1 +at_1^3}$$
$$\mathbf{t_2 x + y = 2at_2 +at_2^3}$$
$$\mathbf{t_3 x + y = 2at_3 +at_3^3}$$
\noindent
\section*{\textbf{Solution}}
\noindent
Considering coefficients of three lines in matrix form :
\begin{align}
\myvec{t_1&1}\vec{x}=2at_1+at_1^3\\
\myvec{t_2&1}\vec{x}=2at_2+at_2^3\\
\myvec{t_3&1}\vec{x}=2at_3+at_3^3
\end{align}
The above equations form a matrix equation as below:
\begin{align}
\myvec{t_1&1\\t_2&1\\t_3&1}\vec{x}=\myvec{2at_1+at_1^3\\2at_2+at_2^3\\2at_3+at_3^3}
\end{align}
Given the lines are concurrent ,so considering above equations are consistent and are reduced to augmented form as below to find the condition for lines to be concurrent:
\begin{align}
\myvec{
t_1&1&-2at_1-at_1^3\\
t_2&1&-2at_2-at_2^3\\
t_3&1&-2at_3-at_3^3
}
\end{align}
Considering Transpose of above augmented form of matrix,we get:
\myvec{t_1&t_2&t_3\\1&1&1\\-2at_1-at_1^3&-2at_2-at_2^3&-2at_3-at_3^3
}
Performing row operations on the above augmented matrix as follows:
\begin{align}
\begin{split}
\myvec{t_1&t_2&t_3\\1&1&1\\-2at_1-at_1^3&-2at_2-at_2^3&-2at_3-at_3^3
}\\
\xleftrightarrow{R_3\leftarrow {R_3+2aR_1}}\\  
\myvec{t_1&t_2&t_3\\1&1&1\\-at_1^3&-at_2^3&-at_3^3}\\
\xleftrightarrow{R_3 \leftarrow {R_3/(-a)}}\\
\myvec{t_1&t_2&t_3\\1&1&1\\t_1^3&t_2^3&t_3^3}
\end{split}
\end{align}
Since system of equations are considered consistent,we get
\begin{align}
t_1(t_3^3-t_2^3)-t_2(t_3^3-t_1^3)+t_3(t_2^3-t_1^3)=0\\
t_1(t_2^3-t_3^3)+t_2(t_3^3-t_1^3)+t_3(t_1^3-t_2^3)=0\\
t_1t_2^3-t_1t_3^3+t_2t_3^3-t_2t_1^3+t_3t_1^3-t_3t_2^3=0
\end{align}
\begin{equation}
t_1t_2(t_2^2-t_1^2)+t_2t_3(t_3^2-t_2^2)+t_3t_1(t_1^2-t_3^2)=0
\label{eq:1}
\end{equation}
Simplifying further Equation \eqref{eq:1} we get 
\begin{equation}
\begin{split}
(t_1t_2(t_2-t_1)(t_2+t_1))+(t_2t_3(t_3-t_2)(t_3+t_2))\\
+(t_3t_1(t_1-t_3)(t_1+t_3))=0
\end{split}
\label{eq:2}
\end{equation}
Therefore Equation \eqref{eq:2} represents the condition for the three lines to be concurrent.
\end{document}

